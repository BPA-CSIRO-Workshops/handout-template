%
% Example Workshop Handout with title page and a section demonstrating
% the different environments defined in the top matter
%
\documentclass[a4paper,12pt,twoside]{memoir}
\usepackage{btp}    % Use the trainermanual package option (i.e. \usepackage[trainermanual]{btp}) to generate the Trainer's version of the manual

% Set some Workshop specific info
\setWorkshopTitle{My Workshop}
\setWorkshopVenue{Somewhere, Earth}
\setWorkshopDate{Feb 2013}
\setWorkshopAuthor{Me\\
Myself\\
and I}


\begin{document}

%
% Workshop Title Page
%
\workshoptitlepage

\section{Example Section Heading}

\begin{information}
Information to be provided to the trainee.
\end{information}

\begin{steps}
Instructions for the trainee to perform.
\end{steps}

\begin{note}
Something of note.
\end{note}

\begin{warning}
A warning to the trainee which needs to be read carefully.
\end{warning}

\begin{questions}
First question.
\begin{answer}
Answer to first question.
\end{answer}

Second question.
\begin{answer}
Answer to second question.
\end{answer}
\end{questions}

\begin{bonus}
An optional bonus section for those progressing rapidly.
\end{bonus}

\begin{advanced}
An optional advanced section for those progressing very rapidly or to be used for future reference.
\end{advanced}

\begin{lstlisting}
# several lines of code
cd ~/
ls -l
# a long command that line wraps automatically
tophat --solexa-quals -g 2 --library-type fr-unstranded -j annotation/Danio_rerio.Zv9.66.spliceSites -o tophat/ZV9_2cells genome/ZV9 data/2cells_1.fastq data/2cells_2.fastq
\end{lstlisting}

\end{document}
